\documentclass[10pt, a4paper, landscape]{extarticle}

% -----packages-----
\usepackage{multicol} % for multiple columns
\usepackage[landscape]{geometry} % for landscape
\usepackage{parskip} % remove text indentation
\usepackage{graphicx} % for scale tables
\usepackage{enumitem} % indent of lists
\usepackage{tikz} % for plots
\usepackage{hyperref} % for hyperlinks
\usepackage{amsmath} % for writing normal text on equations
\usepackage{scrlayer-scrpage} % page foot
\usepackage[compact]{titlesec} % titles spacing

%%%%%%%%%%%%%%%%%%%%%%%%%%%%%%%%%%%%%%%%%%%%%%
%%%%%%%% WATERMARK (NOT PERMANENT...) %%%%%%%%
%%%%%%%%%%%%%%%%%%%%%%%%%%%%%%%%%%%%%%%%%%%%%%
\usepackage{draftwatermark}
%%%%%%%%%%%%%%%%%%%%%%%%%%%%%%%%%%%%%%%%%%%%%%

% -----page customization-----
\geometry{top=1cm,left=1cm,right=1cm,bottom=1cm} % margins configuration
\pagenumbering{gobble} % remove page numeration
\setlength{\parskip}{0cm} % paragraph skip length
% title spacing:
\titlespacing{\section}{0pt}{2ex}{1ex}
\titlespacing{\subsection}{0pt}{1ex}{0ex}
\titlespacing{\subsubsection}{0pt}{0.5ex}{0ex}

%%%%%%%%%%%%%%%%%%%%%%%%%%%%%%%%%%%%%%
%%%%%%%% WATERMARK PROPERTIES %%%%%%%%
%%%%%%%%%%%%%%%%%%%%%%%%%%%%%%%%%%%%%%
\SetWatermarkText{DRAFT}
\SetWatermarkScale{3}
%%%%%%%%%%%%%%%%%%%%%%%%%%%%%%%%%%%%%%

% -----document-----
\begin{document}

% page foot
\cfoot{\href{https://github.com/marcelomijas/econometrics-cheatsheet}{\normalfont \footnotesize Version pr0.3-en - github.com/marcelomijas/econometrics-cheatsheet}}
\setlength{\footskip}{12pt}

\begin{multicols}{3} % set columns to 3

\begin{center}
	\textbf{\LARGE \href{https://github.com/marcelomijas/econometrics-cheatsheet}{Problems Cheat Sheet}}
	\\ {\footnotesize By Marcelo Moreno - King Juan Carlos University}
	\\ {\footnotesize As part of the Econometrics Cheat Sheet Project}
\end{center}

\colorbox{yellow}{THIS IS A WORK IN PROGRESS}
\\ \colorbox{yellow}{NOT INTENDEND FOR GENEAL PURPOSE}

\section*{Multicollinearity}
	\begin{itemize}[leftmargin=*]
		\item \textbf{Perfect multicollinearity} - there are independent variables that are constant and/or there is an exact linear relation between independent variables. Is the \textbf{breaking of the third (3) econometric model assumption}.
		\item \textbf{Approximate multicollinearity} - there are independent variables that are approximately constant and/or there is an approximately linear relation between independent variables. It \textbf{does not break any econometric model assumption}, but has an effect on OLS.
	\end{itemize}
	\subsection*{Consequences}
		\begin{itemize}[leftmargin=*]
			\item \textbf{Perfect multicollinearity} - the equation system of OLS cannot be solved due to infinite solutions.
			\item \textbf{Approximate multicollinearity}
			\begin{itemize}[leftmargin=*]
				\item Small sample variations can induce to big variations in the OLS estimations.
				\item The variance of the OLS estimators of the $x$'s that are collinear, increments, thus the inference of the parameter is affected. The estimation of the parameter is very imprecise (big confidence interval).
			\end{itemize}
		\end{itemize}
	\subsection*{Detection}
		\begin{itemize}[leftmargin=*]
			\item \textbf{Correlation analysis} - look for high correlations (greater than 0.7) between independent variables.
			\item \textbf{Variance Inflation Factor (VIF)} - indicates the increment of $Var(\hat{\beta}_j)$ because of the multicollinearity.
			\begin{center}
				$VIF(\hat{\beta}_j) = \frac{1}{1-R_j^2}$
			\end{center}
			Where $R^2_j$ denotes the r-squared from a regression between $x_j$ and all the other $x$'s. 
			\begin{itemize}[leftmargin=*]
				\item Values between 4 to 10 suggest that it is advisable to analyze in more depth if there might be multicollinearity problems.
				\item Values bigger than 10 indicates that there are multicollinearity problems.
			\end{itemize}
		\end{itemize}
		One typical characteristic of multicollinearity is that the regression coefficients of the model are not individually different from zero (due to high variances), but jointly they are different from zero.
	\subsection*{Correction}
		\begin{itemize}[leftmargin=*]
			\item Delete one of the collinear variables.
			\item Perform factorial analysis (or any other dimension reduction technique) on the collinear variables.
			\item Interpret coefficients with multicollinearity jointly.
		\end{itemize}

\columnbreak

\section*{Heteroscedasticity}
	The residuals $u_i$ of the population regression function do not have the same variance $\sigma^2$:
	\begin{center}
		$Var(u|x) = Var(y|x) \neq \sigma^2$
	\end{center}
	Is the \textbf{breaking of the fifth (5) econometric model assumption}.
	\subsection*{Consequences}
		\begin{itemize}[leftmargin=*]
			\item OLS estimators still are unbiased.
			\item OLS estimators still are consistent.
			\item OLS is \textbf{not efficient} anymore, but still a LUE (Linear Unbiased Estimator).
			\item \textbf{Variance estimations of the estimators are biased}: the construction of confidence intervals and the hypothesis contrast are not reliable.
		\end{itemize}
	\subsection*{Detection}
		\begin{itemize}[leftmargin=*]
			\setlength{\multicolsep}{0pt} % reduce vertical spacing betwen subsection and multicols
			\setlength{\columnsep}{20pt} % increment spacing between columns
			\begin{multicols}{3} % set columns to 3
			\item \textbf{Graphs} - look for scatter patterns on $x$ vs. $u$ or $x$ vs. $y$ plots.
			
			\columnbreak
			\vspace*{-23pt}
			
			\begin{tikzpicture}[scale=0.108]
				\draw[step=2, gray, very thin] (-10,-10) grid (10,10); % grid
				\draw[thick,->] (-10,0) -- (10,0) node[anchor=north] {$x$}; % x axis
				\draw[thick,-] (-10,-10) -- (-10,10) node[anchor=west] {$u$}; % u axis
				\draw plot [only marks, mark=*, mark size=6, domain=0:17, samples=50] ({\x - 9},{-0.5 * rand * \x - 1}); % data points
				\draw[thick, dashed, red, -latex] plot [domain=1:17] ({\x - 10},{-0.5 * \x - 1}); % lower red arrow
				\draw[thick, dashed, red, -latex] plot [domain=1:17] ({\x - 10},{0.5 * \x - 1}); % upper red arrow
			\end{tikzpicture}
			
			\columnbreak
			\vspace*{-24pt}

			\begin{tikzpicture}[scale=0.108]
				\draw[step=2, gray, very thin] (-10,-10) grid (10,10); % grid
				\draw[thick,<->] (-10,10) node[anchor=west] {$y$} -- (-10,-10) -- (10,-10) node[anchor=south] {$x$}; % axis
				\draw plot [only marks, mark=*, mark size=6, domain=0:13, samples=50] ({\x -9},{(-0.65 * rand * \x) + 0.6 * \x - 8}); % data points
				\draw[thick, dashed, red, -latex] plot [domain=0:12] ({\x - 9},{-0.06 * \x - 8.5}); % lower red arrow
				\draw[thick, dashed, red, -latex] plot [domain=0:12] ({\x -9},{1.25 * \x - 7.2}); % upper red arrow
			\end{tikzpicture}
			\end{multicols}
			\item \textbf{Formal tests} - White, Bartlett, Breusch-Pagan, etc. Commonly, the null hypothesis: $H_0 = \text{Homoscedasticity}$
		\end{itemize}
	\subsection*{Correction}
		\begin{itemize}[leftmargin=*]
			\item Use OLS with a variance-covariance matrix estimator robust to heteroscedasticity, for example, the one proposed by White.
			\item If the variance structure is known, make use of Weighted Least Squares (WLS) or Generalized Least Squares (GLS).
			\item If the variance structure is not known, make use of Feasible Weighted Least Squared (FWLS), that estimates a possible variance, divides the model variables by it and then apply OLS.
			\item Make assumptions about the possible variance:
			\begin{itemize}[leftmargin=*]
				\item Supposing that $\sigma_i^2$ is proportional to $x_i$, divide the model variables by the square root of $x_i$ and apply OLS.
				\item Supposing that $\sigma_i^2$ is proportional to $x_i^2$, divide the model variables by $x_i$ and apply OLS.
			\end{itemize}
			\item Make a new model specification, for example, logarithmic transformation.
		\end{itemize}

\columnbreak

\section*{Auto-correlation}
	The residual of any observation, $u_t$, is correlated with the residual of any other observation. The observations are not independent.
	\begin{center}
		$Corr(u_t, u_s | x) \neq 0$ for any $t \neq s$
	\end{center}
	The ``natural" context of this phenomena is time series. Is the \textbf{breaking of the sixth (6) econometric model assumption}.
	\subsection*{Consequences}
		\begin{itemize}[leftmargin=*]
			\item OLS estimators still are unbiased.
			\item OLS estimators still are consistent.
			\item OLS is \textbf{not efficient} anymore, but still a LUE (Linear Unbiased Estimator).
			\item \textbf{Variance estimations of the estimators are biased}: the construction of confidence intervals and the hypothesis contrast are not reliable.
		\end{itemize}
	\subsection*{Detection}
		\begin{itemize}[leftmargin=*]
			\item \textbf{Graphs} - look for scatter patterns on $u_{t-1}$ vs. $u_t$ or make use of a correlogram.
			\setlength{\multicolsep}{0pt} % reduce vertical spacing betwen subsection and multicols
			\setlength{\columnsep}{6pt} % increment spacing between columns
			\begin{multicols}{3} % set columns to 3
				\begin{center}
					\textbf{\footnotesize AR}
				\end{center}
				\vspace{2.0pt}
				\begin{tikzpicture}[scale=0.11]
					\draw[step=2, gray, very thin] (-10,-10) grid (10,10); % grid
					\draw[thick,->] (-10,0) -- (10,0) node[anchor=south] {$u_{t-1}$}; % ut-1 axis
					\draw[thick,-] (-10,-10) -- (-10,10) node[anchor=west] {$u_t$}; % ut axis
					\draw plot [only marks, mark=*, mark size=6, domain=-8:8, samples=50] (\x,{rnd * 6 + (-2 * (\x)^2 + 40) * 0.1}); % data points
					\draw[thick, dashed, red, -latex] plot [domain=-8:8] (\x,{3 + (-2 * (\x)^2 + 40) * 0.1}); % red arrow
				\end{tikzpicture}

			\columnbreak

				\begin{center}
					\textbf{\footnotesize AR(+)}
				\end{center}
				\begin{tikzpicture}[scale=0.11]
					\draw[step=2, gray, very thin] (-10,-10) grid (10,10); % grid
					\draw[thick,->] (-10,0) -- (10,0) node[anchor=north] {$ u_{t-1}$}; % ut-1 axis
					\draw[thick,-] (-10,-10) -- (-10,10) node[anchor=west] {$u_t$}; % ut axis
					\draw plot [only marks, mark=*, mark size=6, domain=-8:8, samples=25] (\x,{rnd * 6 + 0.5 * \x - 3}); % data points
					\draw[thick, dashed, red, -latex] plot [domain=-8:8] (\x,{3 + 0.5 * \x - 3}); % red arrow
				\end{tikzpicture}

			\columnbreak
			
				\begin{center}
					\textbf{\footnotesize AR(-)}
				\end{center}
				\begin{tikzpicture}[scale=0.11]
					\draw[step=2, gray, very thin] (-10,-10) grid (10,10); % grid
					\draw[thick,->] (-10,0) -- (10,0) node[anchor=south] {$u_{t-1}$}; % ut-1 axis
					\draw[thick,-] (-10,-10) -- (-10,10) node[anchor=west] {$u_t$}; % ut axis
					\draw plot [only marks, mark=*, mark size=6, domain=-8:8, samples=25] (\x,{rnd * 6 - 0.5 * \x - 3}); % data points
					\draw[thick, dashed, red, -latex] plot [domain=-8:8] (\x,{3 - 0.5 * \x - 3}); % red arrow
				\end{tikzpicture}
			\end{multicols}
			\item \textbf{Formal tests} - Durbin-Watson, Breusch-Godfrey, etc. Commonly, the null hypothesis: $H_0: \text{No auto-correlation}$
		\end{itemize}
	\subsection*{Correction}
		\begin{itemize}[leftmargin=*]
			\item Use OLS with a variance-covariance matrix estimator robust to auto-correlation, for example, the one proposed by Newey-West.
			\item Use Generalized Least Squares. Supposing $y_t = \beta_0 + \beta_1 x_t + u_t$, with $u_t = \rho u_{t-1} + \varepsilon_t$, where $|\rho| < 1$ and $\varepsilon_t$ is white noise.
			\begin{itemize}[leftmargin=*]
				\item If $\rho$ is known, create a quasi-differentiated model where $u_t$ is white noise and estimate it by OLS.
				\item If $\rho$ is not known, estimate it by -for example- the Cochrane-Orcutt method, create a quasi-differentiated model where $u_t$ is white noise and estimate it by OLS.
			\end{itemize}
		\end{itemize}

\end{multicols}

\end{document}