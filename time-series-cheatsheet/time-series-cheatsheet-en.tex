\documentclass[10pt, a4paper, landscape]{extarticle}

% ----- packages -----
\usepackage{amsmath, amsfonts, amssymb} % better math
\usepackage{enumitem} % better lists
\usepackage{geometry} % margins
\usepackage{graphicx} % scaling
\usepackage{hyperref} % hyperlinks
\usepackage{multicol} % multiple columns
\usepackage{parskip} % paragraph spacing
\usepackage{scrlayer-scrpage} % page foot
\usepackage{tikz} % plots
\usepackage{titlesec} % titles
\usetikzlibrary{patterns} % patterns

% ----- random seed -----
\pgfmathsetseed{12}

% ----- custom commands -----
\newcommand{\E}{\mathrm{E}}
\newcommand{\Var}{\mathrm{Var}}
\newcommand{\se}{\mathrm{se}}
\newcommand{\Cov}{\mathrm{Cov}}
\newcommand{\Corr}{\mathrm{Corr}}

% ----- page customization -----
\geometry{margin=1cm} % margins config
\pagenumbering{gobble} % remove page numeration
\setlength{\parskip}{0cm} % paragraph spacing
% title spacing
\titlespacing{\section}{0pt}{2ex}{1ex}
\titlespacing{\subsection}{0pt}{1ex}{0ex}
\titlespacing{\subsubsection}{0pt}{0.5ex}{0ex}

% ----- document -----
\begin{document}

\cfoot{\href{https://github.com/marcelomijas/econometrics-cheatsheet}{\normalfont \footnotesize TS-24.5-EN - github.com/marcelomijas/econometrics-cheatsheet - CC-BY-4.0 license}}
\setlength{\footskip}{12pt}

\begin{multicols}{3}

\begin{center}
	\textbf{\LARGE \href{https://github.com/marcelomijas/econometrics-cheatsheet}{Time Series Cheat Sheet}}

	{\footnotesize By Marcelo Moreno - Universidad Rey Juan Carlos}

	{\footnotesize The Econometrics Cheat Sheet Project}
\end{center}

\section*{Basic concepts}

\subsection*{Definitions}

\textbf{Time series} - is a succession of quantitative observations of a phenomena ordered in time.

There are some variations of time series:

\begin{itemize}[leftmargin=*]
	\item \textbf{Panel data} - consist of a time series for each observation of a cross section.
	\item \textbf{Pooled cross sections} - combines cross sections from different time periods.
\end{itemize}

\textbf{Stochastic process} - sequence of random variables that are indexed in time.

\subsection*{Components of a time series}

\begin{itemize}[leftmargin=*]
	\item \textbf{Trend} - is the long-term general movement of a series.
	\item \textbf{Seasonal variations} - are periodic oscillations that are produced in a period equal or inferior to a year, and can be easily identified on different years (usually are the result of climatology reasons).
	\item \textbf{Cyclical variations} - are periodic oscillations that are produced in a period greater than a year (are the result of the economic cycle).
	\item \textbf{Residual variations} - are movements that do not follow a recognizable periodic oscillation (are the result of eventual non-permanent phenomena that can affect the studied variable in a given moment).
\end{itemize}

\subsection*{Type of time series models}

\begin{itemize}[leftmargin=*]
	\item \textbf{Static models} - the relation between $y$ and $x$'s is contemporary. Conceptually:
	\begin{center}
		$y_t = \beta_0 + \beta_1 x_t + u_t$
	\end{center}
	\item \textbf{Distributed-lag models} - the relation between $y$ and $x$'s is not contemporary. Conceptually:
	\begin{center}
		$y_t = \beta_0 + \beta_1 x_t + \beta_2 x_{t - 1} + \cdots + \beta_{s} x_{t - (s - 1)} + u_t$
	\end{center}
	The long term cumulative effect in $y$ when $\Delta x$ is:
	\begin{center}
	 	$\beta_1 + \beta_2 + \cdots + \beta_{s}$
	\end{center}
 	\item \textbf{Dynamic models} - a temporal drift of the dependent variable is part of the independent variables (endogeneity). Conceptually:
 	\begin{center}
 		$y_t = \beta_0 + \beta_1 y_{t - 1} + \cdots + \beta_s y_{t - s} + u_t$
 	\end{center}
	\item Combinations of the above, like the rational distributed-lag models (distributed-lag + dynamic).
\end{itemize}

\columnbreak

\section*{Assumptions and properties}

\subsection*{OLS model assumptions under time series}

Under this assumptions, the OLS estimator will present good properties. \textbf{Gauss-Markov assumptions} extended applied to time series:

\begin{enumerate}[leftmargin=*, label=t\arabic*.]
	\item \textbf{Parameters linearity and weak dependence}.
	\begin{enumerate}[leftmargin=*, label=\alph*.]
		\item $y_t$ must be a linear function of the $\beta$'s.
		\item The stochastic $\lbrace(x_t, y_t) : t = 1, 2, \ldots, T\rbrace$ is stationary and weakly dependent.
	\end{enumerate}
	\item \textbf{No perfect collinearity}.
	\begin{itemize}[leftmargin=*]
		\item There are no independent variables that are constant: $\Var(x_j) \neq 0, \; \forall j = 1, \ldots, k$
		\item There is not an exact linear relation between independent variables.
	\end{itemize}
	\item \textbf{Conditional mean zero and correlation zero}.
	\begin{enumerate}[leftmargin=*, label=\alph*.]
		\item There are no systematic errors: $\E(u \mid x_1, \ldots, x_k) = \E(u) = 0 \rightarrow$ \textbf{strong exogeneity} (a implies b).
		\item There are no relevant variables left out of the model: $\Cov(x_j , u) = 0, \; \forall j = 1, \ldots, k \rightarrow$ \textbf{weak exogeneity}.
	\end{enumerate}
	\item \textbf{Homoscedasticity}. The variability of the residuals is the same for any $x$: $\Var(u \mid x_1, \ldots, x_k) = \sigma^2_u$
	\item \textbf{No auto-correlation}. Residuals do not contain information about any other residuals: \\ $\Corr(u_t, u_s \mid x_1, \ldots, x_k) = 0, \; \forall t \neq s$
	\item \textbf{Normality}. Residuals are independent and identically distributed (\textbf{i.i.d.} so on): $u \sim \mathcal{N} (0, \sigma^2_u)$
	\item \textbf{Data size}. The number of observations available must be greater than $(k + 1)$ parameters to estimate. (It is already satisfied under asymptotic situations)
\end{enumerate}

\subsection*{Asymptotic properties of OLS}

Under the econometric model assumptions and the Central Limit Theorem:

\begin{itemize}[leftmargin=*]
	\item Hold t1 to t3a: OLS is \textbf{unbiased}. $\E(\hat{\beta}_j) = \beta_j$
	\item Hold t1 to t3: OLS is \textbf{consistent}. $\mathrm{plim}(\hat{\beta}_j) = \beta_j$ (to t3b left out t3a, weak exogeneity, biased but consistent)
	\item Hold t1 to t5: \textbf{asymptotic normality} of OLS (then, t6 is necessarily satisfied): $u \underset{a}{\sim} \mathcal{N} (0, \sigma^2_u)$
	\item Hold t1 to t5: \textbf{unbiased estimate} of $\sigma^2_u$. $\E(\hat{\sigma}^2_u) = \sigma^2_u$
	\item Hold t1 to t5: OLS is \textcolor{blue}{BLUE} (Best Linear Unbiased Estimator) or \textbf{efficient}. 
	\item Hold t1 to t6: hypothesis testing and confidence intervals can be done reliably.
\end{itemize}

\columnbreak

\section*{Trends and seasonality}

\textbf{Spurious regression} - is when the relation between $y$ and $x$ is due to factors that affect $y$ and have correlation with $x$, $\Corr(x_j, u) \neq 0$. Is the \textbf{non-fulfillment of t3}.

\subsection*{Trends}

Two time series can have the same (or contrary) trend, that should lend to a high level of correlation. This, can provoke a false appearance of causality, the problem is \textbf{spurious regression}. Given the model:

\begin{center}
	$y_t = \beta_0 + \beta_1 x_t + u_t$
\end{center}

where:

\begin{center}
	$y_t = \alpha_0 + \alpha_1 \mathrm{Trend} + v_t$

	$x_t = \gamma_0 + \gamma_1 \mathrm{Trend} + v_t$
\end{center}

Adding a trend to the model can solve the problem:

\begin{center}
	$y_t = \beta_0 + \beta_1 x_t + \beta_2 \mathrm{Trend} + u_t$
\end{center}

The trend can be linear or non-linear (quadratic, cubic, exponential, etc.)

Another way is make use of the \textbf{Hodrick-Prescott filter} to extract the trend and the cyclical component.

\subsection*{Seasonality}

\setlength{\multicolsep}{0pt}
\begin{multicols}{2}

	A time series with can manifest seasonality. That is, the series is subject to a seasonal variations or pattern, usually related to climatology conditions.
	
	For example, GDP (black) is usually higher in summer and lower in winter. Seasonally adjusted series ({\color{red} red}) for comparison.

\columnbreak

	\begin{tikzpicture}[scale=0.18]
		% \draw [step=2, gray, very thin] (-10, -10) grid (10, 10); % grid
		\draw [thick, <->] (-10, 10) node [anchor=south] {$y$} -- (-10, -10) -- (10, -10) node [anchor=north] {$t$}; %axis
		\draw [thick, black]
		(-10.0, -7.206) -- 
		(-9.5, -5.190) -- 
		(-9.0, -7.500) -- 
		(-8.5, -2.381) -- 
		(-8.0, -3.969) -- 
		(-7.5, -1.160) -- 
		(-7.0, -4.580) -- 
		(-6.5, 0.855) -- 
		(-6.0, -1.526) -- 
		(-5.5, -0.305) -- 
		(-5.0, -4.519) -- 
		(-4.5, -0.488) -- 
		(-4.0, -2.320) -- 
		(-3.5, -0.427) -- 
		(-3.0, -4.213) -- 
		(-2.5, 0.366) -- 
		(-2.0, -1.709) -- 
		(-1.5, -0.549) -- 
		(-1.0, -4.396) -- 
		(-0.5, 1.099) -- 
		(0.0, -1.038) -- 
		(0.5, 1.282) -- 
		(1.0, -2.870) -- 
		(1.5, 1.709) -- 
		(2.0, -1.038) -- 
		(2.5, 1.526) -- 
		(3.0, -1.832) -- 
		(3.5, 3.358) -- 
		(4.0, 1.099) -- 
		(4.5, 4.213) -- 
		(5.0, 0.916) -- 
		(5.5, 6.290) -- 
		(6.0, 4.396) -- 
		(6.5, 6.595) -- 
		(7.0, 3.419) -- 
		(7.5, 8.000) -- 
		(8.0, 6.106) -- 
		(8.5, 6.900);
		\draw [thick, red] 
		(-10.0, -6.2061) -- 
		(-9.5, -6.0018) -- 
		(-9.0, -6.1000) -- 
		(-8.5, -5.0817) -- 
		(-8.0, -3.9095) -- 
		(-7.5, -3.0603) -- 
		(-7.0, -3.0002) -- 
		(-6.5, -2.4550) -- 
		(-6.0, -2.5267) -- 
		(-5.5, -2.3053) -- 
		(-5.0, -2.5191) -- 
		(-4.5, -2.4885) -- 
		(-4.0, -2.3206) -- 
		(-3.5,  -2.4275) -- 
		(-3.0, -2.2137) -- 
		(-2.5, -1.6664) -- 
		(-2.0, -2.0099) -- 
		(-1.5, -1.8496) -- 
		(-1.0, -1.3969) -- 
		(-0.5, -1.0992) -- 
		(0.0, -1.0382) -- 
		(0.5, -1.2824) -- 
		(1.0, -1.0002) -- 
		(1.5, -0.8099) -- 
		(2.0, -0.6382) -- 
		(2.5, -0.6267) -- 
		(3.0, 0.6321) -- 
		(3.5, 1.0588) -- 
		(4.0, 1.3992) -- 
		(4.5, 2.2137) -- 
		(5.0, 2.5160) -- 
		(5.5, 3.5901) -- 
		(6.0, 4.0969) -- 
		(6.5, 5.2954) -- 
		(7.0, 5.3198) -- 
		(7.5, 6.2000) -- 
		(8.0, 6.9069) -- 
		(8.5, 7.2008);
	\end{tikzpicture}

\end{multicols}

\begin{itemize}[leftmargin=*]
	\item This problem is \textbf{spurious regression}. A seasonal adjustment can solve it.
\end{itemize}

A simple \textbf{seasonal adjustment} could be creating stationary binary variables and adding them to the model. For example, for quarterly series ($Qq_t$ are binary variables):

\begin{center}
	$y_t = \beta_0 + \beta_1 Q2_t + \beta_2 Q3_t + \beta_3 Q4_t + \beta_4 x_{1t} + \cdots + \beta_k x_{kt} + u_t$
\end{center}

Another way is to seasonally adjust (sa) the variables, and then, do the regression with the adjusted variables:

\begin{center}
	$z_t = \beta_0 + \beta_1 Q2_t + \beta_2 Q3_t + \beta_3 Q4_t  + v_t \rightarrow \hat{v}_t + \E(z_t) = \hat{z}_t^{sa}$

	$\hat{y}_t^{sa} = \beta_0 + \beta_1 \hat{x}_{1t}^{sa} + \cdots + \beta_k \hat{x}_{kt}^{sa} + u_t$
\end{center}

There are much better and complex methods to seasonally adjust a time series, like the \textbf{X-13ARIMA-SEATS}.

\columnbreak

\section*{Auto-correlation}

The residual of any observation, $u_t$, is correlated with the residual of any other observation. The observations are not independent. Is the \textbf{non-fulfillment} of \textbf{t5}.

\begin{center}
	$\Corr(u_t, u_s \mid x_1, \ldots, x_k) = \Corr(u_t, u_s) \neq 0, \; \forall t \neq s$
\end{center}

\subsection*{Consequences}

\begin{itemize}[leftmargin=*]
	\item OLS estimators still are unbiased.
	\item OLS estimators still are consistent.
	\item OLS is \textbf{not efficient} anymore, but still a LUE (Linear Unbiased Estimator).
	\item \textbf{Variance estimations} of the estimators are \textbf{biased}: the construction of confidence intervals and the hypothesis testing is not reliable.
\end{itemize}

\subsection*{Detection}

\begin{itemize}[leftmargin=*]
	\item \textbf{Scatter plots} - look for scatter patterns on $u_{t - 1}$ vs. $u_t$.
	\setlength{\multicolsep}{0pt}
	\setlength{\columnsep}{6pt}
	\begin{multicols}{3}
		\begin{center}
			\begin{tikzpicture}[scale=0.11]
				\node at (0, 13) {\textbf{Ac.}};
				% \draw [step=2, gray, very thin] (-10, -10) grid (10, 10); % grid
				\draw [thick, ->] (-10, 0) -- (10, 0) node [anchor=south] {$u_{t - 1}$}; % ut-1 axis
				\draw [thick, -] (-10, -10) -- (-10, 10) node [anchor=west] {$u_t$}; % ut axis
				\draw plot [only marks, mark=*, mark size=6, domain=-8:8, samples=50] (\x, {rnd*6 + (-2*(\x)^2 + 40)*0.1}); % data points
				\draw [thick, dashed, red, -latex] plot [domain=-8:8] (\x, {3 + (-2*(\x)^2 + 40)*0.1}); % red arrow
			\end{tikzpicture}
		\end{center}

	\columnbreak

		\begin{center}
			\begin{tikzpicture}[scale=0.11]
				\node at (0, 13) {\textbf{Ac.(+)}};
				% \draw [step=2, gray, very thin] (-10, -10) grid (10, 10); % grid
				\draw [thick, ->] (-10, 0) -- (10, 0) node [anchor=north] {$ u_{t - 1}$}; % ut-1 axis
				\draw [thick, -] (-10, -10) -- (-10, 10) node [anchor=west] {$u_t$}; % ut axis
				\draw plot [only marks, mark=*, mark size=6, domain=-8:8, samples=25] (\x, {rnd*6 + 0.5*\x - 3}); % data points
				\draw [thick, dashed, red, -latex] plot [domain=-8:8] (\x, {3 + 0.5*\x - 3}); % red arrow
			\end{tikzpicture}
		\end{center}

	\columnbreak

		\begin{center}
			\begin{tikzpicture}[scale=0.11]
				\node at (0, 13) {\textbf{Ac.(-)}};
				% \draw [step=2, gray, very thin] (-10, -10) grid (10, 10); % grid
				\draw [thick, ->] (-10, 0) -- (10, 0) node [anchor=south] {$u_{t - 1}$}; % ut-1 axis
				\draw [thick, -] (-10, -10) -- (-10, 10) node [anchor=west] {$u_t$}; % ut axis
				\draw plot [only marks, mark=*, mark size=6, domain=-8:8, samples=25] (\x, {rnd*6 - 0.5*\x - 3}); % data points
				\draw [thick, dashed, red, -latex] plot [domain=-8:8] (\x, {3 - 0.5*\x - 3}); % red arrow
			\end{tikzpicture}
		\end{center}

	\end{multicols}

	\begin{multicols}{2}

		\item \textbf{Correlogram} - composed of the auto-correlation function (ACF) and the partial ACF (PACF).

	\columnbreak

		\begin{itemize}[leftmargin=*]
			\item Y axis: correlation [-1, 1].
			\item X axis: lag number.
			\item Blue lines: $\pm 1.96/T^{0.5}$
		\end{itemize}
	\end{multicols}

	\begin{center}
		\begin{tikzpicture}[scale=0.22]
			% acf plot
			\node at (17.5, 23) {\tiny \textbf{ACF}};
			\node at (-1.5, 22) {\tiny 1};
			\node at (-1.5, 17) {\tiny 0};
			\node at (-1.5, 12) {\tiny -1};
			\draw [step=2, gray, thin] (0, 12) rectangle (35, 22);
			\draw [thick, |->] (0, 22) -- (0, 12) -- (35, 12);
			\fill [red] (1.5, 17) rectangle (2, 21.9);
			\fill [red] (3.5, 17) rectangle (4, 21.7);
			\fill [red] (5.5, 17) rectangle (6, 21.5);
			\fill [red] (7.5, 17) rectangle (8, 21.2);
			\fill [red] (9.5, 17) rectangle (10, 20.8);
			\fill [red] (11.5, 17) rectangle (12, 20.6);
			\fill [red] (13.5, 17) rectangle (14, 20.4);
			\fill [red] (15.5, 17) rectangle (16, 20.3);
			\fill [red] (17.5, 17) rectangle (18, 20.2);
			\fill [red] (19.5, 17) rectangle (20, 20.1);
			\fill [red] (21.5, 17) rectangle (22, 20.0);
			\fill [red] (23.5, 17) rectangle (24, 19.9);
			\fill [red] (25.5, 17) rectangle (26, 19.8);
			\fill [red] (27.5, 17) rectangle (28, 19.7);
			\fill [red] (29.5, 17) rectangle (30, 19.6);
			\fill [red] (31.5, 17) rectangle (32, 19.5);
			\fill [red] (33.5, 17) rectangle (34, 19.3);
			\draw [blue, thin] (0, 18.5) -- (35, 18.5);
			\draw [dashed, thin] (0, 17) -- (35, 17);
			\draw [blue, thin] (0, 15.5) -- (35, 15.5);
			% pacf plot
			\node at (17.5, 11) {\tiny \textbf{PACF}};
			\node at (-1.5, 10) {\tiny 1};
			\node at (-1.5, 5) {\tiny 0};
			\node at (-1.5, 0) {\tiny -1};
			\draw [step=2, gray, thin] (0, 0) rectangle (35, 10);
			\draw [thick, |->] (0, 10) -- (0, 0) -- (35, 0);
			\fill [red] (1.5, 5) rectangle (2, 9.9);
			\fill [red] (3.5, 5) rectangle (4, 3.0);
			\fill [red] (5.5, 5) rectangle (6, 4.5);
			\fill [red] (7.5, 5) rectangle (8, 4.2);
			\fill [red] (9.5, 5) rectangle (10, 5.8);
			\fill [red] (11.5, 5) rectangle (12, 4.7);
			\fill [red] (13.5, 5) rectangle (14, 4.5);
			\fill [red] (15.5, 5) rectangle (16, 4.8);
			\fill [red] (17.5, 5) rectangle (18, 5.6);
			\fill [red] (19.5, 5) rectangle (20, 4.9);
			\fill [red] (21.5, 5) rectangle (22, 4.7);
			\fill [red] (23.5, 5) rectangle (24, 4.6);
			\fill [red] (25.5, 5) rectangle (26, 5.4);
			\fill [red] (27.5, 5) rectangle (28, 4.9);
			\fill [red] (29.5, 5) rectangle (30, 4.8);
			\fill [red] (31.5, 5) rectangle (32, 4.8);
			\fill [red] (33.5, 5) rectangle (34, 5.1);
			\draw [blue, thin] (0, 6.5) -- (35, 6.5);
			\draw [dashed, thin] (0, 5) -- (35, 5);
			\draw [blue, thin] (0, 3.5) -- (35, 3.5);
		\end{tikzpicture}
	\end{center}

	Conclusions differ between auto-correlation processes.

\columnbreak

	\begin{itemize}[leftmargin=*]
		\item \textbf{MA($q$) process}. \textbf{ACF}: only the first $q$ coefficients are significant, the remaining are abruptly canceled. \textbf{PACF}: attenuated exponential fast decay or sine waves.
		\item \textbf{AR($p$) process}. \textbf{ACF}: attenuated exponential fast decay or sine waves. \textbf{PACF}: only the first $p$ coefficients are significant, the remaining are abruptly canceled.
		\item \textbf{ARMA($p, q$) process}. \textbf{ACF} and \textbf{PACF}: the coefficients are not abruptly canceled and presents a fast decay.
	\end{itemize}

	If the ACF coefficients do not decay rapidly, there is a clear indicator of lack of stationarity in mean, which would lead to take first differences in the	original series.

	\item \textbf{Formal tests} - Generally, $H_0$: No auto-correlation.

	Supposing that $u_t$ follows an AR(1) process:
	
	\begin{center}
		$u_t = \rho_1 u_{t - 1} + \varepsilon_t$
	\end{center}

	where $\varepsilon_t$ is white noise.
	
	\begin{itemize}[leftmargin=*]
		\item \textbf{AR(1) t test} (exogenous regressors):
		\begin{center}
			$t = \frac{\hat{\rho}_1}{\se(\hat{\rho}_1)} \sim t_{T - k - 1, \alpha/2}$
		\end{center}
		\begin{itemize}[leftmargin=*]
			\item $H_1$: Auto-correlation of order one, AR(1).
		\end{itemize}
		\item \textbf{Durbin-Watson statistic} (exogenous regressors and residual normality):
		\begin{center}
			$d = \frac{\sum_{t=2}^n (\hat{u}_t - \hat{u}_{t - 1})^2}{\sum_{t=1}^n \hat{u}_t^2} \approx 2 \cdot (1 - \hat{\rho}_1), \; 0 \leq d \leq 4$
		\end{center}
		\begin{itemize}[leftmargin=*]
			\item $H_1$: Auto-correlation of order one, AR(1).
		\end{itemize}
		\begin{center}
			\scalebox{0.8}{
				\begin{tabular}{| c | c | c | c |}
					\hline
					$d =$          & 0 & 2 & 4  \\ \hline
					$\rho \approx$ & 1 & 0 & -1 \\ \hline
				\end{tabular}
			}
			\begin{tikzpicture}[scale=0.28]
				\fill [pattern=north east lines, pattern color=gray] (5, 0) rectangle  (9, 9);
				\draw (5, 0) -- (5, 9);
				\draw (9, 0) -- (9, 9);
				\fill [pattern=north east lines, pattern color=gray] (16, 0) rectangle (20, 9);
				\draw (16, 0) -- (16, 9);
				\draw (20, 0) -- (20, 9);
				\draw [thick] (0, 9) -- (0, 0) -- (25, 0);
				\draw [dashed] (12.5, 0) -- (12.5, 9);
				\node at (-1.5, 8.5) {\scalebox{1.2}{\tiny $f(d)$}};
				\node at (0, -0.6) {\scalebox{1.2}{\tiny 0}};
				\node at (5, -0.6) {\scalebox{1.2}{\tiny $d_L$}};
				\node at (9, -0.6) {\scalebox{1.2}{\tiny $d_U$}};
				\node at (12.5, -0.6) {\scalebox{1.2}{\tiny 2}};
				\node at (16.7, -1) {\scalebox{1.1}{\tiny \rotatebox{-20}{$4 - d_U$}}};
				\node at (20.7, -1) {\scalebox{1.1}{\tiny \rotatebox{-20}{$4 - d_L$}}};
				\node at (25, -0.6) {\scalebox{1.2}{\tiny 4}};
				\node at (2.5, 5.5) {\scalebox{1.2}{\tiny Rej. $H_0$}};
				\node at (2.5, 4.5) {\scalebox{1.2}{\tiny AR(+)}};
				\node [text=red] at (7, 4.5) {\scalebox{1.2}{\tiny \rotatebox{-70}{\textbf{INCONCLUSIVE}}}};
				\node at (12.5, 5.5) {\scalebox{1.2}{\tiny Accept $H_0$}};
				\node at (12.5, 4.5) {\scalebox{1.2}{\tiny No AR}};
				\node [text=red] at (18, 4.5) {\scalebox{1.2}{\tiny \rotatebox{-70}{\textbf{INCONCLUSIVE}}}};
				\node at (22.5, 5.5) {\scalebox{1.2}{\tiny Rej. $H_0$}};
				\node at (22.5, 4.5) {\scalebox{1.2}{\tiny AR(-)}};
			\end{tikzpicture}
		\end{center}
		\item \textbf{Durbin's h} (endogenous regressors):
		\begin{center}
			$h = \hat{\rho} \cdot \sqrt{\frac{T}{1 - T \cdot \upsilon}}$
		\end{center}
		where $\upsilon$ is the estimated variance of the coefficient associated to the endogenous variable.
		\begin{itemize}[leftmargin=*]
			\item $H_1$: Auto-correlation of order one, AR(1).
		\end{itemize}
		\item \textbf{Breusch-Godfrey test} (endogenous regressors): it can detect MA($q$) and AR($p$) processes ($\varepsilon_t$ is w. noise):
		\begin{itemize}[leftmargin=*]
			\item MA($q$): $u_t = \varepsilon_t - m_1 u_{t - 1} - \cdots - m_q u_{t - q}$
			\item AR($p$): $u_t = \rho_1 u_{t - 1} + \cdots + \rho_p u_{t - p} + \varepsilon_t$
		\end{itemize}
		Under $H_0$: No auto-correlation:
		\begin{center}
			$\hfill T \cdot R^2_{\hat{u}_t} \underset{a}{\sim} \chi^2_q \hfill \textbf{or} \hfill T \cdot R^2_{\hat{u}_t} \underset{a}{\sim} \chi^2_p \hfill$
		\end{center}
		\begin{itemize}[leftmargin=*]
			\item $H_1$: Auto-correlation of order $q$ (or $p$).
		\end{itemize}
		\item \textbf{Ljung-Box Q test}:
		\begin{itemize}[leftmargin=*]
			\item $H_1$: There is auto-correlation.
		\end{itemize}
	\end{itemize}

\end{itemize}

\subsection*{Correction}

\begin{itemize}[leftmargin=*]
	\item Use OLS with a variance-covariance matrix estimator that is \textbf{robust to heterocedasticity and auto-correlation} (HAC), for example, the one proposed by \textbf{Newey-West}.
	\item Use \textbf{Generalized Least Squares} (GLS). Supposing $y_t = \beta_0 + \beta_1 x_t + u_t$, with $u_t = \rho u_{t - 1} + \varepsilon_t$, where $\lvert \rho \rvert < 1$ and $\varepsilon_t$ is \underline{white noise}.
	\begin{itemize}[leftmargin=*]
		\item If $\rho$ is \textbf{known}, use a \textbf{quasi-differentiated model}:
		\begin{center}
			$y_t - \rho y_{t - 1} = \beta_0 (1 - \rho) + \beta_1 (x_t - \rho x_{t - 1}) + u_t - \rho u_{t - 1}$
			
			$y_t^* = \beta_0^* + \beta_1' x_t^* + \varepsilon_t$
		\end{center}
		where $\beta_1' = \beta_1$; and estimate it by OLS.
		\item If $\rho$ is \textbf{not known}, estimate it by -for example- the \textbf{Cochrane-Orcutt iterative method} (Prais-Winsten method is also good):
		\begin{enumerate}[leftmargin=*]
			\item Obtain $\hat{u}_t$ from the original model.
			\item Estimate $\hat{u}_t = \rho \hat{u}_{t-1} + \varepsilon_t$ and obtain $\hat{\rho}$.
			\item Create a quasi-differentiated model:
			\begin{center}
				$y_t - \hat{\rho} y_{t - 1} = \beta_0 (1 - \hat{\rho}) + \beta_1 (x_t - \hat{\rho} x_{t - 1}) + u_t - \hat{\rho} u_{t - 1}$
				
				$y_t^* = \beta_0^* + \beta_1' x_t^* + \varepsilon_t$
			\end{center}
			where $\beta_1' = \beta_1$; and estimate it by OLS.
			\item Obtain $\hat{u}_t^* = y_t - (\hat{\beta}_0^* + \hat{\beta}_1' x_t) \neq y_t - (\hat{\beta}_0^* + \hat{\beta}_1' x_t^*)$.
			\item Repeat from step 2. The algorithm ends when the estimated parameters vary very little between iterations.
		\end{enumerate}
	\end{itemize}
	\item If not solved, look for \textbf{high dependence} in the series.
\end{itemize}

\section*{Stationarity and weak dependence}

Stationarity means stability of the joint distributions of a process as time progresses. It allows to correctly identify the relations --that stay unchange with time-- between variables.

\subsection*{Stationary and non-stationary processes}

\begin{itemize}[leftmargin=*]
	\item \textbf{Stationary process} (strong stationarity) - is the one in that the probability distributions are stable in time: if any collection of random variables is taken, and then, shifted $h$ periods, the joint probability distribution should stay unchanged. It's easier to analyze and model.

\columnbreak

	\item \textbf{Non-stationary process} - is, for example, a series with trend, where at least the mean changes with time.
	\item \textbf{Covariance stationary process} - is a weaker form of stationarity:
	\begin{itemize}[leftmargin=*]
		\item $\E(x_t)$ is constant.
		\item $\Var(x_t)$ is constant.
		\item For any $t$,  $h \geq 1$, the $\Cov(x_t, x_{t + h})$ depends only of $h$, not of $t$.
	\end{itemize}
\end{itemize}

\subsection*{Weakly dependent time series}

It is important because it replaces the random sampling assumption, giving for granted the validity of the Central Limit Theorem (requires stationarity and a form of weak dependence). Weakly dependent processes are also known as \textbf{integrated of order zero}, I(0).

\begin{itemize}[leftmargin=*]
	\item \textbf{Weakly dependent} - restricts how close the relationship between $x_t$ and $x_{t + h}$ can be as the time distance between the series increases ($h$).
\end{itemize}

An \textbf{stationary time process} $\lbrace x_t : t = 1, 2, \ldots, T \rbrace$ is weakly dependent when $x_t$ and $x_{t + h}$ are almost independent as $h$ increases without a limit.

A \textbf{covariance stationary time process} is weakly dependent if the correlation between $x_t$ and $x_{t + h}$ tends to $0$ fast enough when $h \rightarrow \infty$ (they are not asymptotically correlated).

Some examples of stationary and weakly dependent time series are:

\begin{itemize}[leftmargin=*]
	\item \textbf{Moving average} - $\lbrace x_t \rbrace$ is a moving average of order one MA($q$):
	\begin{center}
		$x_t = e_t + m_1 e_{t - 1} + \cdots + m_q e_{t - q}$
	\end{center}
	where $\lbrace e_t : t = 0, 1, \ldots, T \rbrace$ is an \textsl{i.i.d.} sequence with zero mean and $\sigma^2_e$ variance.
	\item \textbf{Auto-regressive process} - $\lbrace x_t \rbrace$ is an auto-regressive process of order one AR($p$):
	\begin{center}
		$x_t = \rho_1 x_{t - 1} + \cdots + \rho_p x_{t - p} + e_t$
	\end{center}
	where $\lbrace e_t : t = 1, 2, \ldots, T \rbrace$ is an \textsl{i.i.d.} sequence with zero mean and $\sigma^2_e$ variance.

	If $\lvert \rho_1 \rvert < 1$, then $\lbrace x_t \rbrace$ is an AR(1) stable process that is weakly dependent. It is stationary in covariance, $\Corr(x_t, x_{t - 1}) = \rho_1$.
	\item \textbf{ARMA process} - is a combination of the two above. $\lbrace x_t \rbrace$ is an ARMA($p, q$):
	\begin{center}
		$x_t = e_t + m_1 e_{t - 1} + \cdots + m_q e_{t - q} + \rho_1 x_{t - 1} + \cdots + \rho_p x_{t - p}$
	\end{center}
\end{itemize}

A series with a trend cannot be stationary, but can be weakly dependent (and stationary if the series is de-trended).

\columnbreak

\section*{Strongly dependent time series}

Most of the time, economics series are strongly dependent (or high persistent in time). Some special cases of \textbf{unit root} processes, I(1):

\begin{itemize}[leftmargin=*]
		\item \textbf{Random walk} - an AR(1) process with $\rho_1 = 1$.
	\begin{center}
		$y_t = y_{t - 1} + e_t$
	\end{center}
	where $\lbrace e_t : t = 1, 2, \ldots, T \rbrace$ is an \textsl{i.i.d.} sequence with zero mean and $\sigma^2_e$ variance.
	
	\item \textbf{Random walk with a drift} - an AR(1) process with $\rho_1 = 1$ and a constant.
	\begin{center}
		$y_t = \beta_0 + y_{t - 1} + e_t$
	\end{center}
	where $\lbrace e_t : t = 1, 2, \ldots, T \rbrace$ is an \textsl{i.i.d.} sequence with zero mean and $\sigma^2_e$ variance.
\end{itemize}

\subsection*{I(1) detection}

\begin{itemize}[leftmargin=*]
	\item \textbf{Augmented Dickey-Fuller} (ADF) \textbf{test} - where $H_0$: the process is unit root, I(1).
	\item \textbf{Kwiatkowski–Phillips–Schmidt–Shin} (KPSS) \textbf{test} - where $H_0$: the process have no unit root, I(0).
	\item \textbf{Phillips-Perron} (PP) \textbf{test} - where $H_0$: the process is unit root, I(1).
\end{itemize}

\subsection*{Transforming unit root to weak dependent}

\textbf{Unit root} processes are \textbf{integrated of order one}, I(1). This means that \textbf{the first difference} of the process is \textbf{weakly dependent} or I(0) (and usually, stationary). For example, a random walk:

\begin{multicols}{2}

\begin{center}
	$\Delta y_t = y_t - y_{t - 1} = e_t$
\end{center}

where $\lbrace e_t \rbrace = \lbrace \Delta y_t \rbrace$  is \textsl{i.i.d.} \\

Getting the first difference of a series also deletes its trend. \\

For example, a series with a trend (black), and it's first difference ({\color{red} red}).

\columnbreak

\begin{tikzpicture}[scale=0.18]
	% \draw [step=2, gray, very thin] (-10, -10) grid (10, 10); % grid
	\draw [thick, <->] (-10, 10) node [anchor=south west] {$y, {\color{red} \Delta y}$} -- (-10, -10) -- (10, -10) node [anchor=north] {$t$}; %axis
	\draw [thick, black]
	(-10.0, -8.000) -- 
	(-9.5, -7.541) -- 
	(-9.0, -7.284) -- 
	(-8.5, -6.795) -- 
	(-8.0, -6.429) -- 
	(-7.5, -6.048) -- 
	(-7.0, -5.953) -- 
	(-6.5, -5.486) -- 
	(-6.0, -5.281) -- 
	(-5.5, -4.840) -- 
	(-5.0, -4.326) -- 
	(-4.5, -4.013) -- 
	(-4.0, -3.758) -- 
	(-3.5, -3.529) -- 
	(-3.0, -3.056) -- 
	(-2.5, -2.896) -- 
	(-2.0, -2.416) -- 
	(-1.5, -1.913) -- 
	(-1.0, -1.888) -- 
	(-0.5, -1.166) -- 
	(0.0, -0.530) -- 
	(0.5, -0.282) -- 
	(1.0, 0.032) -- 
	(1.5, 0.491) -- 
	(2.0, 0.748) -- 
	(2.5, 0.805) -- 
	(3.0, 1.016) -- 
	(3.5, 1.439) -- 
	(4.0, 1.810) -- 
	(4.5, 2.247) -- 
	(5.0, 2.668) -- 
	(5.5, 3.052) -- 
	(6.0, 3.586) -- 
	(6.5, 4.322) -- 
	(7.0, 4.913) -- 
	(7.5, 5.704) -- 
	(8.0, 6.081) -- 
	(8.5, 6.431);
	\draw [thick, red]
	(-9.5, 1.283) -- 
	(-9.0, -2.799) -- 
	(-8.5, 1.889) -- 
	(-8.0, -0.595) -- 
	(-7.5, -0.299) -- 
	(-7.0, -6.074) -- 
	(-6.5, 1.454) -- 
	(-6.0, -3.864) -- 
	(-5.5, 0.926) -- 
	(-5.0, 2.393) -- 
	(-4.5, -1.655) -- 
	(-4.0, -2.843) -- 
	(-3.5, -3.373) -- 
	(-3.0, 1.572) -- 
	(-2.5, -4.765) -- 
	(-2.0, 1.703) -- 
	(-1.5, 2.186) -- 
	(-1.0, -7.487) -- 
	(-0.5, 6.607) -- 
	(0.0, 4.869) -- 
	(0.5, -2.985) -- 
	(1.0, -1.632) -- 
	(1.5, 1.283) -- 
	(2.0, -2.804) -- 
	(2.5, -6.847) -- 
	(3.0, -3.723) -- 
	(3.5, 0.547) -- 
	(4.0, -0.483) -- 
	(4.5, 0.834) -- 
	(5.0, 0.526) -- 
	(5.5, -0.246) -- 
	(6.0, 2.816) -- 
	(6.5, 6.875) -- 
	(7.0, 3.961) -- 
	(7.5, 8.000) -- 
	(8.0, -0.356) -- 
	(8.5, -0.925);
\end{tikzpicture}

\end{multicols}

When an I(1) series is strictly positive, it is usually converted to logarithms before taking the first difference. That is, to obtain the (approx.) percentage change of the series:

\begin{center}
	$\Delta \log(y_t) = \log(y_t) - \log(y_{t - 1}) \approx \dfrac{y_t - y_{t - 1}}{y_{t - 1}}$
\end{center}

\columnbreak

\section*{Cointegration}

When \textbf{two series are I(1), but a linear combination of them is I(0)}. If the case, the regression of one series over the other is not spurious, but expresses something about the long term relation. Variables are called cointegrated if they have a common stochastic trend.

For example: $\lbrace x_t \rbrace$ and $\lbrace y_t \rbrace$ are I(1), but $y_t - \beta x_t = u_t$ where $\lbrace u_t \rbrace$ is I(0). ($\beta$ get the name of cointegration parameter).

\section*{Heterocedasticity on time series}

The \textbf{assumption} affected is \textbf{t4}, which leads \textbf{OLS to be not efficient}.

Some tests that work could be the Breusch-Pagan or White's, where $H_0$: No heterocedasticity. It is \textbf{important} for the tests to work that there is \textbf{no auto-correlation} (so first, it is imperative to test for it).

\subsection*{ARCH}

An auto-regressive conditional heterocedasticity (ARCH), is a model to analyze a form of dynamic heterocedasticity, where the error variance follows an AR($p$) process.

Given the model: $y_t = \beta_0 + \beta_1 z_t + u_t$ where, there is AR(1) and heterocedasticity:

\begin{center}
	$\E(u^2_t \mid u_{t - 1}) = \alpha_0 + \alpha_1 u^2_{t - 1}$
\end{center}

\subsection*{GARCH}

A general auto-regressive conditional heterocedasticity (GARCH), is a model similar to ARCH, but in this case, the error variance follows an ARMA($p, q$) process.

\section*{Exponential smoothing}

\begin{center}
	$f_t = \alpha y_t + (1 - \alpha) f_{t - 1}$
\end{center}

where $0 < \alpha < 1$ is the smoothing parameter.

\section*{Predictions}

Two types of prediction:

\begin{itemize}[leftmargin=*]
	\item Of the mean value of $y$ for a specific value of $x$.
	\item Of an individual value of $y$ for a specific value of $x$.
\end{itemize}

If the values of the variables ($x$) approximate to the mean values ($\overline{x}$), the confidence interval amplitude of the prediction will be shorter.

\end{multicols}

\end{document}